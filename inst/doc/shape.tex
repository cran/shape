\documentclass[article,nojss]{jss}
\DeclareGraphicsExtensions{.pdf,.eps}

%%%%%%%%%%%%%%%%%%%%%%%%%%%%%% Add-on packages and fonts
\usepackage{amsmath}
\usepackage{xspace}
\usepackage{verbatim}
\usepackage[english]{babel}
%\usepackage{mathptmx}
%\usepackage{helvet}
\usepackage[T1]{fontenc}
\usepackage[latin1]{inputenc}

%%%%%%%%%%%%%%%%%%%%%%%%%%%%%% User specified LaTeX commands.
\newcommand{\di}{\textbf{\textsf{diagram}}\xspace}

\title{\proglang{R} Package \pkg{shape}: functions for plotting graphical shapes, colors...}

\Plaintitle{R Package shape: functions for plotting graphical shapes, colors...}

\Keywords{graphics, shapes, colors, R}

\Plainkeywords{graphics, shapes, colors, R}


\author{Karline Soetaert\\
Centre for Estuarine and Marine Ecology\\
Netherlands Institute of Ecology\\
The Netherlands
}

\Plainauthor{Karline Soetaert}

\Abstract{This document describes how to use the \pkg{shape} package
for plotting graphical shapes.

Together with R-package \pkg{diagram} \citep{Soetaert08d} this package has
been written to produce the figures of the book \citep{Soetaert08}

 }

%% The address of (at least) one author should be given
%% in the following format:
\Address{
  Karline Soetaert\\
  Centre for Estuarine and Marine Ecology (CEME)\\
  Netherlands Institute of Ecology (NIOO)\\
  4401 NT Yerseke, Netherlands
  E-mail: \email{k.soetaert@nioo.knaw.nl}\\
  URL: \url{http://www.nioo.knaw.nl/ppages/ksoetaert}\\
}


%%%%%%%%%%%%%%%%%%%%%%%%%%%%%% R/Sweave specific LaTeX commands.
%% need no \usepackage{Sweave}
%\VignetteIndexEntry{shape}

%%%%%%%%%%%%%%%%%%%%%%%%%%%%%% Begin of the document
\begin{document}



\maketitle

\section{Introduction}

This vignette is nothing but a Sweave application of parts of demo \code{colorshapes} in package
\pkg{shape}.


\section{colors}
Although one can find similar functions in other packages (including the R base package),
\pkg{shape} includes ways to generate color schemes; \code{intpalette} creates transitions
between several colors; \code{shadepalette} creates a gradient between two colors,
useful for shading (see below). \code{drapecol} drapes colors over a \code{persp} plot;
by default the red-blue-yellow (matlab-type) colors are used. The code below demonstrates
these functions (Figure \ref{fig:s1})
\begin{Schunk}
\begin{Sinput}
> par(mfrow=c(2,2))
> image(matrix(nrow=1,ncol=50,data=1:50),main="intpalette",
+       col=intpalette(c("red","blue","yellow","green","black"),numcol=50))
> #
> shadepalette(n=10,"white","black")
\end{Sinput}
\begin{Soutput}
 [1] "#000000" "#1C1C1C" "#393939" "#555555" "#717171" "#8E8E8E" "#AAAAAA" "#C6C6C6"
 [9] "#E3E3E3" "#FFFFFF"
\end{Soutput}
\begin{Sinput}
> #
> image(matrix(nrow=1,ncol=50,data=1:50),col=shadepalette(50,"red","blue"),
+        main="shadepalette")
> #
> par(mar=c(0,0,0,0))
> persp(volcano,theta = 135, phi = 30, col = drapecol(volcano),
+        main="drapecol",border=NA)
\end{Sinput}
\end{Schunk}
\begin{figure}
\begin{center}
\begin{Schunk}
\begin{Soutput}
 [1] "#000000" "#1C1C1C" "#393939" "#555555" "#717171" "#8E8E8E" "#AAAAAA" "#C6C6C6"
 [9] "#E3E3E3" "#FFFFFF"
\end{Soutput}
\end{Schunk}
\includegraphics{shape-s1}
\end{center}
\caption{Use of \code{intpalette}, \code{shadepalette} and \code{drapecol}}
\label{fig:s1}
\end{figure}

\section{Rotating}
Function \code{rotatexy} rotates graphical shapes; it can be used to generate
strangely-colored shapes (Figure \ref{fig:s2}).
\begin{Schunk}
\begin{Sinput}
> par(mfrow=c(2,2),mar=c(3,3,3,3))
> #
> # rotating points on a line
> #
> xy <- matrix(ncol=2,data=c(1:5,rep(1,5)))
> plot(xy,xlim=c(-6,6),ylim=c(-6,6),type="b",pch=16,main="rotatexy",col=1)
> for ( i in 1:5) points(rotatexy(xy,mid=c(0,0),angle=60*i),col=i+1,type="b",pch=16)
> points(0,0,cex=2,pch=22,bg="black")
> legend("topright",legend=60*(0:5),col=1:6,pch=16,title="angle")
> legend("topleft",legend="midpoint",pt.bg="black",pt.cex=2,pch=22,box.lty=0)
> #
> # rotating lines..
> #
> x <- seq(0,2*pi,pi/20)
> y <- sin(x)
> cols <- intpalette(c("blue","green","yellow","red"),n=125)
> cols <- c(cols,rev(cols))
> plot(x,y,type="l",ylim=c(-3,3),main="rotatexy",col=cols[1],lwd=2,xlim=c(-1,7))
> for (i in 2:250) lines(rotatexy(cbind(x,y),angle=0.72*i),col=cols[i],lwd=2)
> #
> #
> x <- seq(0,2*pi,pi/20)
> y <- sin(x*2)
> cols <- intpalette(c("red","yellow","black"),n=125)
> cols <- c(cols,rev(cols))
> plot(x,y,type="l",ylim=c(-4,5),main="rotatexy, asp=TRUE",col=cols[1],
+      lwd=2,xlim=c(-1,7))
> for (i in 2:250) lines(rotatexy(cbind(x,y),angle=0.72*i,asp=TRUE),col=cols[i],
+      lwd=2)
> #
> # rotating points
> #
> cols <- femmecol(500)
> plot(x,y,xlim=c(-1,1),ylim=c(-1,1),main="rotatexy",col=cols[1],type="n")
> for (i in 2:500) {xy<-rotatexy(c(0,1),angle=0.72*i, mid=c(0,0));
+                     points(xy[1],xy[2],col=cols[i],pch=".",cex=2)}
\end{Sinput}
\end{Schunk}
\begin{figure}
\begin{center}
\includegraphics{shape-s2}
\end{center}
\caption{Four examples of \code{rotatexy}}
\label{fig:s2}
\end{figure}

\section{ellipses}
If a suitable shading color is used, function \code{filledellipse} creates spheres,
ellipses, donuts with 3-D appearance (Figure \ref{fig:s3}).
\begin{Schunk}
\begin{Sinput}
> par(mfrow=c(2,2),mar=c(2,2,2,2))
> emptyplot(c(-1,1))
> col  <- c(rev(greycol(n=30)),greycol(30))
> filledellipse(rx1=1,rx2=0.5,dr=0.1,col=col)
> title("filledellipse")
> #
> emptyplot(c(-1,1),c(-1,1))
> filledellipse(col=col,dr=0.1)
> title("filledellipse")
> #
> color <-gray(seq(1,0.3,length.out=30))
> emptyplot(xlim=c(-2,2),ylim=c(-2,2),col=color[length(color)])
> filledellipse(rx1=2,ry1=0.4,col=color,angle=45,dr=0.1)
> filledellipse(rx1=2,ry1=0.4,col=color,angle=-45,dr=0.1)
> filledellipse(rx1=2,ry1=0.4,col=color,angle=0,dr=0.1)
> filledellipse(rx1=2,ry1=0.4,col=color,angle=90,dr=0.1)
> title("filledellipse")
> #
> emptyplot(main="getellipse")
> col <-femmecol(90)
> for (i in seq(0,180,by=2))
+     lines(getellipse(0.5,0.25,mid=c(0.5,0.5),angle=i,dr=0.1),
+           type="l",col=col[(i/2)+1],lwd=2)
\end{Sinput}
\end{Schunk}
\begin{figure}
\begin{center}
\includegraphics{shape-s3}
\end{center}
\caption{Use of \code{filledellipse}, and \code{getellipse}}
\label{fig:s3}
\end{figure}

\section{Cylinders, rectangles, multigonals}
The code below draws cylinders, rectangles and multigonals (Figure \ref{fig:s4}).
\begin{Schunk}
\begin{Sinput}
> par(mfrow=c(2,2),mar=c(2,2,2,2))
> #
> # simple cylinders
> emptyplot(c(-1.2,1.2),c(-1,1),main="filledcylinder")
> col  <- c(rev(greycol(n=20)),greycol(n=20))
> col2 <- shadepalette("red","blue",n=20)
> col3 <- shadepalette("yellow","black",n=20)
> filledcylinder(rx=0.,ry=0.2,len=0.25,angle=0,col=col,mid=c(-1,0),dr=0.1)
> filledcylinder(rx=0.0,ry=0.2,angle=90,col=col,mid=c(-0.5,0),dr=0.1)
> filledcylinder(rx=0.1,ry=0.2,angle=90,col=c(col2,rev(col2)),
+                mid=c(0.45,0),topcol=col2[10],dr=0.1)
> filledcylinder(rx=0.05,ry=0.2,angle=90,col=c(col3,rev(col3)),
+                mid=c(0.9,0),topcol=col3[10],dr=0.1)
> filledcylinder(rx=0.1,ry=0.2,angle=90,col="white",lcol="black",
+                lcolint="grey",dr=0.1)
> #
> # more complex cylinders
> emptyplot(c(-1,1),c(-1,1),main="filledcylinder")
> col  <- shadepalette("blue","black",n=20)
> col2 <- shadepalette("red","black",n=20)
> col3 <- shadepalette("yellow","black",n=20)
> filledcylinder(rx=0.025,ry=0.2,angle=90,col=c(col2,rev(col2)),dr=0.1,
+                mid=c(-0.8,0),topcol=col2[10],delt=-1.,lcol="black")
> filledcylinder(rx=0.1,ry=0.2,angle=00,col=c(col,rev(col)),dr=0.1,
+                mid=c(0.0,0.0),topcol=col,delt=-1.2,lcol="black")
> filledcylinder(rx=0.075,ry=0.2,angle=90,col=c(col3,rev(col3)),dr=0.1,
+                mid=c(0.8,0),topcol=col3[10],delt=0.0,lcol="black")
> #
> # rectangles
> color <-shadepalette(grey(0.3),"blue",n=20)
> emptyplot(c(-1,1),main="filledrectangle")
> filledrectangle(wx=0.5,wy=0.5,col=color,mid=c(0,0),angle=0)
> filledrectangle(wx=0.5,wy=0.5,col=color,mid=c(0.5,0.5),angle=90)
> filledrectangle(wx=0.5,wy=0.5,col=color,mid=c(-0.5,-0.5),angle=-90)
> filledrectangle(wx=0.5,wy=0.5,col=color,mid=c(0.5,-0.5),angle=180)
> filledrectangle(wx=0.5,wy=0.5,col=color,mid=c(-0.5,0.5),angle=270)
> #
> # multigonal
> color <-shadepalette(grey(0.3),"blue",n=20)
> emptyplot(c(-1,1))
> filledmultigonal(rx=0.25,ry=0.25,col=shadepalette(grey(0.3),"blue",n=20),
+                  nr=3,mid=c(0,0),angle=0)
> filledmultigonal(rx=0.25,ry=0.25,col=shadepalette(grey(0.3),"darkgreen",n=20),
+                  nr=4,mid=c(0.5,0.5),angle=90)
> filledmultigonal(rx=0.25,ry=0.25,col=shadepalette(grey(0.3),"orange",n=20),
+                  nr=5,mid=c(-0.5,-0.5),angle=-90)
> filledmultigonal(rx=0.25,ry=0.25,col="black",nr=6,mid=c(0.5,-0.5),angle=180)
> filledmultigonal(rx=0.25,ry=0.25,col="white",lcol="black",nr=7,mid=c(-0.5,0.5),
+                  angle=270)
> title("filledmultigonal")
\end{Sinput}
\end{Schunk}
\begin{figure}
\begin{center}
\includegraphics{shape-s4}
\end{center}
\caption{Use of \code{filledcylinder}, \code{filledrectangle} and \code{filledmultigonal}}
\label{fig:s4}
\end{figure}
\section{Other shapes}
Function \code{filledshape} is the most flexible drawing function from \pkg{shape}: just specify an inner and outer shape and
fill with a color scheme (Figure \ref{fig:s5}).
\begin{Schunk}
\begin{Sinput}
> par(mfrow=c(2,2),mar=c(2,2,2,2))
> #an egg
> color <-greycol(30)
> emptyplot(c(-3.2,3.2),col=color[length(color)],main="filledshape")
> b<-4
> a<-9
> x      <- seq(-sqrt(a),sqrt(a),by=0.1)
> g      <- b-b/a*x^2-0.2*b*x+0.2*b/a*x^3
> g[g<0] <- 0
> x1     <-c(x,rev(x))
> g1     <-c(sqrt(g),rev(-sqrt(g)))
> xouter <-cbind(x1,g1)
> xouter <-rbind(xouter,xouter[1,])
> filledshape(xouter,xyinner=c(-1,0),col=color)
> #
> # a mill
> color <-shadepalette(grey(0.3),"yellow",n=20)
> emptyplot(c(-3.3,3.3),col=color[length(color)],main="filledshape")
> x <- seq(0,0.8*pi,pi/20)
> y <- sin(x)
> xouter <- cbind(x,y)
> for (i in seq(0,360,60))
+      xouter <- rbind(xouter,rotatexy(cbind(x,y),mid=c(0,0),angle = i))
> filledshape(xouter,c(0,0),col=color)
> #
> # abstract art
> emptyplot(col="darkgrey",main="filledshape")
> filledshape(matrix(nc=2,runif(80)),col="darkblue")
> #
> emptyplot(col="darkgrey",main="filledshape")
> filledshape(matrix(nc=2,runif(80)),col=shadepalette(20,"darkred","darkblue"))
\end{Sinput}
\end{Schunk}
\begin{figure}
\begin{center}
\includegraphics{shape-s5}
\end{center}
\caption{Use of \code{filledshape}}
\label{fig:s5}
\end{figure}

\section{arrows, arrowheads}
As the arrow heads in the R base package are too simple for some applications,
there are some improved arrow heads in \pkg{shape} (Figure \ref{fig:s6}).
\begin{Schunk}
\begin{Sinput}
> par(mfrow=c(2,2),mar=c(2,2,2,2))
> xlim <- c(-5 ,5)
> ylim <- c(-10,10)
> x0<-runif(100,xlim[1],xlim[2])
> y0<-runif(100,ylim[1],ylim[2])
> x1<-x0+runif(100,-2,2)
> y1<-y0+runif(100,-2,2)
> size <- 0.4
> plot(0,type="n",xlim=xlim,ylim=ylim)
> Arrows(x0,y0,x1,y1,arr.length=size,arr.type="triangle",arr.col=rainbow(runif(100,1,20)))
> title("Arrows")
> #
> # arrow heads
> #
> ang  <- runif(100,-360,360)
> plot(0,type="n",xlim=xlim,ylim=ylim)
> Arrowhead(x0,y0,ang,arr.length=size,arr.type="curved",arr.col=rainbow(runif(100,1,20)))
> title("Arrowhead")
> #
> # Lotka-Volterra competition model
> #
> r1    <- 3              # parameters
> r2    <- 2
> K1    <- 1.5
> K2    <- 2
> alf12 <- 1
> alf21 <- 2
> xlim   <- c(0,1.5)
> ylim   <- c(0,2  )
> par(mar=c(5,4,4,2))
> plot  (0, type="l",lwd=3,   # 1st isocline
+        main="Lotka-Volterra competition",
+        xlab="N1",ylab="N2",xlim=xlim,ylim=ylim)
> gx <- seq(0,1.5,len=30)
> gy <- seq(0,2,len=30)
> N  <- as.matrix(expand.grid(x=gx,y=gy))
> dN1 <- r1*N[,1]*(1-(N[,1]+alf12* N[,2])/K1)
> dN2 <- r2*N[,2]*(1-(N[,2]+alf21* N[,1])/K2)
> dt  <- 0.01
> Arrows(N[,1],N[,2],N[,1]+dt*dN1,N[,2]+dt*dN2,arr.len=0.08, lcol="darkblue",arr.type="triangle")
> points(x=c(0,0,1.5,0.5),y=c(0,2,0,1),pch=22,cex=2,bg=c("white","black","black","grey"))
\end{Sinput}
\end{Schunk}
\begin{figure}
\begin{center}
\includegraphics{shape-s6}
\end{center}
\caption{Use of \code{Arrows} and \code{Arrowhead} }
\label{fig:s6}
\end{figure}


\section{And finally}

This vignette was created using Sweave \citep{Leisch02}.

The package is on CRAN, the R-archive website (\citep{R2008})

More examples can be found in the demo's of package \pkg{ecolMod} \citep{Soetaert08e}

\bibliography{vignettes}

\end{document}

